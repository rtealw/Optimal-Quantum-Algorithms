\section{Introduction}

Our aim is to provide a tool with which one can easily identify 
the optimal quantum query complexity as well as corresponding 
query optimal quantum algorithm for a given Boolean function. This
gives researchers an easy 
way to identify problems for which a quantum computer 
is better suited than a classical computer. It has been shown that
there are many functions for which this is the case, such as
search and factoring.

We say that a function is better suited to a quantum computer than a classical computer if a quantum computer could evaluate the function more efficiently. Generally, we would measure this efficiency by runtime, but the runtime of a quantum computer for a given function is difficult to calculate. Instead, we can approximate the runtime with the algorithm's query complexity. To evaluate a function the algorithm must query the inputs to gain information about the given problem and the number of queries required is the query complexity.

For example, if we take a look at the search function, which is a generalization of the single bit OR function, where there is a bit string input and we return a $1$ to indicate that there is a one somewhere in the input bitstring, and a $0$ if there is not. In a classical computer, in the worst case, we would need to examine all input bits to evaluate this function, which would take $n$ queries for a bitstring of length $n$. However, there exists a quantum implementation of the search function that only requires $\sqrt{n}$ queries of the input bitstring \cite{grover1996fast}--- this is the query optimal quantum algorithm. In this case, there is a large advantage in using a quantum computer as opposed to a classical computer. 

While the optimal quantum query complexity of the search function has been shown to be $\sqrt{n}$, and the query optimal quantum algorithm is known, it is clearly of great interest to identify other functions like search where it would be substantially more efficient to use a quantum computer instead of a classical one. To aid in this search, we have implemented an algorithm that takes a Boolean function and returns the optimal quantum query complexity as well as the query optimal quantum algorithm in the form of a span program. We hope that these results can help build more efficient quantum algorithms and help us better understand the advantages of quantum computers.

\begin{comment}
Given

Why quantum computers
Why we care about query complexity

Quantum computers have many advantages over classical computers as
algorithms for a given function can be substantially less time
complex on quantum computers. More specifically, in many of these
examples, such as search and factoring, quantum computers require
significantly fewer queries of the function input. For an
example, a classic search algorithm for a bit string of length
$n$ could take up to $n$ queries to see if a $1$ is present.
However, quantum computers only require $\sqrt{n}$ queries of the
input.

It is then of great interest to identify these functions to
understand the advantages of quantum computers over classical
computers. We implement such a tool for Boolean functions. We
calculate the asymptotic quantum query complexity of a given
function which provides a lower bound of run time for a quantum
computer. Furthermore, for each Boolean function, we can also
provide a query optimal quantum algorithm such that this optimal
algorithm could be implemented to achieve the potential
improvement over classical computers.

\end{comment}