\section{Introduction}

An important area of research in quantum computing is
to determine the problems for which quantum
computers can outperform classical computers.
To avoid technical problems in the comparison of 
computers--the largest
of which is that large-scale quantum computers
currently do not exist--we turn to theoretical
analysis to characterize the efficiency of 
quantum and classical algorithms.

We implement a tool that places an asymptotic lower bound on runtime by calculating the optimal quantum query complexity of a Boolean function.
The query complexity of a function 
is the number of times we must examine the
contents of the input string to determine the correct
output of the function. 
Each query of the given function takes a certain
amount of time, therefore the query
complexity places a lower bound on runtime.
The reverse is not true: the runtime
may asymptotically exceed query complexity
if the algorithm performs substantially more
computations than queries.

To give a concrete example of query complexity,
recall the canonical OR function.
The $n$-bit OR
function returns $0$ if there are only $0$'s in the input
string and $1$ if there are any $1$'s in the input string.
In the worst case, a classical algorithm must check that
every single bit of the input is a $0$ to return $0$.
Therefore the classical query complexity of the $n$-bit OR
function is $n$. 
However, the quantum algorithm Grover's
search can solve the function OR in $\sqrt{n}$ queries
\cite{grover1996fast}. In the case of OR, we conclude that
the quantum algorithm outperforms the classical algorithm.

For an arbitrary function $f$,
Reichardt presents a semidefinite program (SDP)
whose solution corresponds to the optimal quantum query complexity
of $f$ \cite{reichardt2009span}.
In addition, the solution of the same SDP can
be used to construct a span program that, when run
on a quantum computer, meets the optimal quantum query complexity.

Our contribution is an SDP solver
that finds the optimal quantum query complexity
and query optimal quantum algorithm for given Boolean functions.
While the SDP can be solved for arbitrary functions with finite outputs,
we limit the scope of our work to Boolean functions.

Although there are many SDP solver libraries such as CVXOPT and
SDPA for convex optimization problems,
none easily support solving Reichardt's SDP
\cite{cvxopt, SDPA}.
As a result, we develop an implementation that solves the SDP
and includes optimizations specific to our problem formulation.
We first convert Reichardt's form into the standard SDP form
as described by Boyd \cite{boyd2004convex}, which we verify here.
To solve the SDP, we implement Wen et al.'s alternating direction method (ADM)
algorithm in order to exploit the sparsity of our SDP.
We finally optimize the functions and data structures we use
to speed up our program \cite{adm}. 

Our program takes as input a set of bitstrings $D$
and a Boolean function $f: D \rightarrow \{0,1\}$. 
By solving Reichardt's SDP problem with
Wen et al.'s ADM algorithm,
we return the optimal quantum query complexity of $f$
and a quantum algorithm that meets this query complexity.
We hope that our solver proves useful
for researchers constructing optimal quantum algorithms.
