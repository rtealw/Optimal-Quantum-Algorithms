\section{Methodology}\label{sec:method}
To understand Reichardt's formulation and our conversion
to Boyd's standard form, consider as inputs to our tool
the general case with function $f: D \rightarrow E$ where
$D \subseteq {\{0,1\}}^n$ and $E \subset {\{0,1\}}$. From
previous work, we can formulate the problem as in
\cref{eq:reichardtObj} where $f_{\text{bound}}$ represents
the optimal bound for the function $f$ with the input size
$n$ \cite{reichardt2009span}. Let $F$ be the set of $(y,z)$ such that $f(y) \neq f(z)$.
Then the objective function of the SDP is:
\begin{align} \label{eq:reichardtObj} 
    f_{\text{bound}} = M(\X) = \max_{y \in D} \sum_{j \in [n]}
    \bra{y,j}\X\ket{y,j} 
\end{align}
subject to constraints
\begin{align}\label{Eq:reichardtSemi}
    \X \succcurlyeq 0 
\end{align}
and
\begin{align}\label{Eq:reichardtOffDiag}
    \forall (y,z) \in F \sum_{j \in [n]: y_j \ne z_j} 
    \bra{y,j} \X \ket{z, j} = 1.
\end{align}
Our goal is to minimize $M$, and thus $f_{\text{bound}}$ by finding the optimal value of $\X$.

To make use of this definition, it's helpful to convert
the problem into a more canonical form to make use of
ready-made solvers, and algorithms published on the
topic of solving SDP's generally. We covert Reichardt's problem to Boyd's form as we believe it to be a generally recognized standard form. Boyd's definition is shown below \cite{boyd2004convex}:
\begin{align}\label{Eq:boyd_obj}
    M(\X) = \tr(C\X) 
\end{align}
subject to
\begin{align} \label{Eq:boydSemi}
    \X \succcurlyeq 0   
\end{align}
and for $i \in \{1,...,p\}$
\begin{align} \label{Eq:boydTraceCon}
    \tr(A_i \X) = b_i. 
\end{align}
where $A_i$ and $C$ are matrices and $b_i$ is a scalar for each $i$.

Once we are given and the function $f$ and input size
$n$ of interest, we can then define our set of matrices
and scalars $A_i$, $b_i$, and $C$ such that these two
problems are equivalent. We can then use developed algorithms to solve the problem and obtain our optimal bound on quantum query complexity for the function $f$.

We can begin our conversion by examining the formulation of Reichardt's definition. $\X$ contains chunks of size 
$n \times n$, each corresponding to an element of the Cartesian product $D \times D$. Each axis of the matrix is partitioned into $|D| = 2^n$ equal parts such that we have chunks on the interior of the matrix corresponding to ordered pairs of inputs. Each chunk is of size $n \times n$ as mentioned previously so $\X$ is an $n2^n \times n2^n$ matrix. Below we consider $\X$ of the OR function with one bit inputs:
\begin{align}
\X = \begin{blockarray}{ccc}
\qquad & 0 & 1 \\
\begin{block}{c[cc]}
  0 & X_{(0,0)} & X_{(0,1)} \\
  1 & X_{(1,0)} & X_{(1,1)} \\
\end{block}
\end{blockarray}
\end{align}

In this instance, each sub-matrix $X_{(i,j)}$ is a $ 1
\times 1 $ matrix. This is the same construction used for
each input size $n$ where there is a sub-matrix
corresponding to every ordered pair of inputs in $D$.

To mathematically convert Reichardt's form into an
equivalent SDP in standard form, we first convert the
constraints and then the objective function. It's clear
that constraints in \cref{Eq:reichardtSemi} and 
\cref{Eq:boydSemi} are equivalent, simply requiring that
$\X$ is positive semidefinite.

The second constraint in Reichardt's form requires that
given a sub-matrix corresponding to a pair of inputs with
different outputs, the elements on this sub-matrix's
diagonal corresponding to bits that differ in the two
input strings sum to $1$. For each element in $F$, we can
define a matrix $A_i$ that moves the correct elements of
the corresponding sub-matrix's diagonal to the diagonal
of $A_i \X$ and sets other diagonal entries of $A_i \X$
to zero. In this way the trace of $A_i \X$ is equal to
the sum in \cref{Eq:reichardtOffDiag}. Therefore we set
each $b_i$ to $1$ to make the constraints equivalent.


For example, consider a matrix $X$ where
\begin{align}
    X = \left[ \begin{matrix} 1 & 2 \\ 3 & 4 \end{matrix} \right] \nonumber
\end{align}
and suppose we want to put the top right value $2$ on the diagonal.
We could define $A$ as below:

\begin{align}
    A = \left[ \begin{matrix} 0 & 0 \\ 1 & 0 \end{matrix} \right] \nonumber
\end{align}

Therefore,
\begin{align}
    A \cdot X = \left[ \begin{matrix} 2 & 0 \\ 4 & 0 \end{matrix} \right] \nonumber
\end{align}
and subsequently,
\begin{align}
    \tr(A \cdot X) = 2. \nonumber
\end{align}

Through a similar construction we can build our set of
matrices $A_i$ for each element of $F$ where we set $b_i
= 1$. Having now created $A_i$ and $b_i$ for $i = 1,2,..., |F|$, we have defined constraints equivalent to \cref{Eq:reichardtSemi} and eq. \ref{Eq:reichardtOffDiag} thus creating a SDP constraints in Boyd's form that are equivalent to Reichardt's form.

The problem is now that \cref{Eq:boyd_obj}
does not have a maximum in the way \cref{eq:reichardtObj}
requires. Our solution is to convert this maximum to a
linear function by introducing slack variables. We 
construct a new $\mathcal{X}$ from our original $\X$,
\begin{align}
    \Xb =
    \left[
    \begin{matrix}
    \X & 0 \\
    0 & S
    \end{matrix}
    \right] \nonumber
\end{align}
%\begin{equation}
%    \Xb = \left[\begin{matrix} \X \cdots \cdots 0 \\
%                                \vdots \ddots \vdots \\
%                                0 \cdots \cdots z  \end{matrix}
%                                \right]  
%\end{equation}
where $S$ is a diagonal matrix
with objective function $z$ in the bottom right most entry
and slack variables $s_i$ for $i \in [|D|]$
along the rest of the diagonal.

We guarantee that $z$ is the maximum
diagonal chunk of $\X$
by generating another set of matrices 
$\Ap_i$ and scalars $\bp_i$ for $i \in [|D|]$
where $\bp_i = 0$.
For each element of $D$, define a $c_i$ such that
\begin{align}
    c_i = \sum_{j \in [n]} \bra{y,j}\X\ket{y,j}
    \nonumber
\end{align} 
where $y$ is the $i^{th}$ element of $D$, making $c_i$ equivalent to the trace of the corresponding sub-matrix on the diagonal of $\X$.
Our goal is to enforce that $z$ is the maximum $c_i$.
By requiring that $s_i + c_i = z$ and using
the fact that $s_i$ is positive
($\Xb$ is semidefinite so $S$ is as well),
we can keep $z$ larger than or equal to $c_i$.
The semidefinite program minimizes $z$
by our choice of $C$ so $z$ meets the maximum $c_i$.

To enforce $s_i + c_i = z$, define $\Ap_i$ as a diagonal
matrix  with 1s along the chunk
corresponding to the $i^{th}$ input,
1 in the entry corresponding to $s_i$,
and -1 in the entry corresponding to $z$.
In a slight abuse of notation, we show an example
of $\Xb A_i$.
\begin{align}
\left[\begin{matrix} c_i & 0 & 0 \\
                    0 & s_i & 0 \\
                    0 & 0 & z \end{matrix} \right]
\left[\begin{matrix} 1 & 0 & 0 \\
                    0 & 1 & 0 \\
                    0 & 0 & -1 \end{matrix} \right]
= \left[\begin{matrix} c_i & 0 & 0 \\
                    0 & s_i & 0 \\
                    0 & 0 & -z \end{matrix} \right]
            \nonumber
\end{align}
Notice that the trace of $\Xb A_i$
is $c_i + s_i - z$.
By setting $\bp_i = 0$, we have that $z$ is
greater than or equal to $c_i$.

The final matrix we define is $C$.
Our goal is to select out $z$ from $\Xb$
so $C$ has a single non-zero value in the 
entry corresponding to $z$.
\begin{align}
    C = \left[\begin{matrix} 0  \cdots 0 \\
                                \vdots \ddots \vdots \\
                                0 \cdots  1  \end{matrix}
                                \right]  
                                \nonumber
\end{align}

We will now prove that using our construction,
$z = \max\{c_1, c_2, ... c_{|D|}\}$ in the optimal $\Xb$

Notice that because our matrix $\Xb$ is semidefinite 
and diagonal, all of its diagonal entries are non-negative. 
Assume for contradiction that $z < c_i$
for some $i \in \{1,2,..., |D|\}$.
Therefore $z < c_i + s_i$ since $s_i \geq 0$.

By our constraint, $z = c_i + s_i$, 
meaning that $z < z$, a contradiction! 
Therefore it must be the case that $z$ is an 
upper bound of the traces of our sub-matrices $c_i$. 

Now assume for contradiction that $z \ne  \max\{c_1, c_2, ...
c_{|D|}\}$. We know that $z$ is an upper bound so we only need to
show that there is no upper bound of this finite set that is less
than $z$ to show it is the maximum. Our assumption is equivalent to
the statement that there does not exist a constant $a$ such that $a
> c_i \forall i$ and $a < z$. If $a$ did exist, then there would
exist an $\Xb$ that also satisfies all of our constraints,
but with $a$ in the bottom right most entry instead of $z$. This
would mean that our objective function was improved over our
initial matrix as $a < z$, meaning that the $\Xb$ we found
is not optimal: a contradiction. We conclude that our constraints
are equivalent to those outlined by Reichardt, while matching the
form of Boyd.

Once given a function $f$ and all possible inputs to the
function of length $n$ ($D$), we can then define matrices
$C, A_i, \Ap_i$ and vectors $b_i, \bp_i$ such that the solution of Boyd's SDP
produces the solution of the Reichardt's SDP which
produces the optimal bound of quantum query complexity for
arbitrary Boolean functions.