\section{Methodology}\label{sec:method}

Consider the function $f: D \rightarrow E$ where $D
\subseteq {\{0,1\}}^n$ and $E \subset {0,1}$. We aim to find an
optimal bound on quantum query complexity using semidefinite
programming. From previous work,
we can formulate the problem as in \cref{eq:reichardtObj} where
$f_{\text{bound}}$ represents the optimal bound for the function
$f$ with the input size $n$ \cite{reichardt2009span}.
Let $F$ be the set of $(y,z)$ such that $f(y) \neq f(z)$.
Then the objective function of the SDP is
\begin{align} \label{eq:reichardtObj}
    f_{\text{bound}} = M(\X) = \max_{y \in D} \sum_{j \in [n]}
    \bra{y,j}\X\ket{y,j} 
\end{align}
subject to constraints
\begin{align}\label{Eq:semi1}
    \X \succcurlyeq 0 
\end{align}
and
\begin{align}\label{Eq:off-diag}
    \forall (y,z) \in F \sum_{j \in [n]: y_j \ne z_j} 
    \bra{y,j} \X \ket{z, j} = 1.
\end{align}
Our goal is to minimize $M$ by finding the optimal value of $\X$.

We construct the matrices in the SDP definition.
Observe that $\X$ is a $n2^n \times n2^n$ matrix 
because there are $2^n$ possible inputs each of length $n$. 
Think of $\X$ as containing chunks of size 
$n \times n$ where each chunk corresponds to an element of $D$. % x $D$.
Below we consider $\X$ of the OR function with one bit inputs,
\begin{align}
    \X = \left[
    \begin{matrix}
        1 & 1 \\
        1 & 1
    \end{matrix}
    \right]
    \nonumber
\end{align}
%\begin{align}
%   X = \left[ \begin{matrix}
%   \left[ \begin{matrix} (0,0)\end{matrix} \right] & \left[
%   \begin{matrix} (0,1)\end{matrix} \right] \\ \\
%   \left[ \begin{matrix} (1,0)\end{matrix} \right] & \left[
%   \begin{matrix} (1,1)\end{matrix} \right] \\
%    \end{matrix} \right]
%\end{align}

Our objective function \cref{eq:reichardtObj} is
the maximum trace of the diagonal sub-matrices. The
first constraint \cref{semi1} is that $\X$ is positive
semidefinite. The final constraint \cref{semi2} states that
diagonal entries where the input bits differ
of each sub-matrix where the Boolean outputs are not equal must sum to one.

Given this non-traditional formation of the semidefinite programming problem, we first convert the problem into an equivalent problem in standard form as defined by Boyd \cite{boyd2004convex}:
\begin{align}\label{Eq:boyd_obj}
    M(\X) = tr(C\X) 
\end{align}
subject to
\begin{align}
    \X \succcurlyeq 0   
\end{align}
and for $i \in \{1,...,|F|\}$
\begin{align}\label{Eq:trace}
    \tr(A_i \X) = b_i. 
\end{align}


To make these two problems equivalent we first note that
conditions \cref{Eq:semi1} and \cref{Eq:off-diag} are
the same.
To equate conditions \cref{Eq:off-diag}
and \cref{Eq:trace}, we define the set of matrices
$A_i$ and
scalars $b_i$ where $i \in [|F|]$.
We set $b_i$ equal to $1$ to satisfy \cref{Eq:off-diag}.
Observe that the sum in condition \cref{Eq:off-diag}
iterates through a portion of the matrix $\X$ diagonally.
Thus we can define a set of matrices where the matrix 
$A_i$ contains only 1s and 0s 
such that $A_i \X$ puts the off-diagonal values we want
to sum onto the diagonal.
Therefore the trace of this new matrix $A_i \X$ 
will be equal to the sum in \cref{Eq:off-diag}.

For example, consider a matrix $X$ where
\begin{align}
    X = \left[ \begin{matrix} 1 & 2 \\ 3 & 4 \end{matrix} \right] \nonumber
\end{align}
and suppose we want to put the top right value $2$ on the diagonal.
We could define $A$ as below:

\begin{align}
    A = \left[ \begin{matrix} 0 & 0 \\ 1 & 0 \end{matrix} \right] \nonumber
\end{align}

Therefore,
\begin{align}
    A \cdot X = \left[ \begin{matrix} 2 & 0 \\ 4 & 0 \end{matrix} \right] \nonumber
\end{align}
and subsequently,
\begin{align}
    \tr(A \cdot X) = 2. \nonumber
\end{align}

Through a similar construction we can build a set of matrices 
$A_i$ for each element of $F$ such that setting $b_i = 1$ 
creates an equivalent constraint to \cref{Eq:off-diag}.
Having now created $A_i$ and $b_i$ for $i = 1,2,...,
|F|$, we have defined constraints equivalent to eq.
\ref{Eq:semi1} and eq. \ref{Eq:off-diag} thus creating
a SDP constraints in Boyd's form that are equivalent to
Reichardt's form.

Our set of $A_i$ now satisfies \cref{Eq:off-diag}.
The problem is that \cref{Eq:boyd_obj}
does not have a maximum in the way \cref{eq:reichardtObj}
requires.
Our solution is to introduce slack variables
that enforce that our objective function is the maximum.
The big picture is that we construct a new
from our original $\X$,
\begin{align}
    \Xb =
    \left[
    \begin{matrix}
    \X & 0 \\
    0 & S
    \end{matrix}
    \right] \nonumber
\end{align}
%\begin{equation}
%    \Xb = \left[\begin{matrix} \X \cdots \cdots 0 \\
%                                \vdots \ddots \vdots \\
%                                0 \cdots \cdots z  \end{matrix}
%                                \right]  
%\end{equation}
where $S$ is a diagonal matrix
with objective function $z$ on the bottom right
and slack variables $s_i$ for $i \in [|D|]$
along the rest of the diagonal.

We guarantee that $z$ is the maximum
diagonal chunk of $\X$
by generating another set of matrices 
$\Ap_i$ and scalars $b_i$ for $i \in [|D|]$
where $b_i = 0$.
For each element of $D$, define a $c_i$ such that
\begin{align}
    c_i = \sum_{j \in [n]} \bra{y,j}\X\ket{y,j}
    \nonumber
\end{align} 
where $y$ is the $i^{th}$ element of $D$.
Our goal is to enforce that $z$ is the maximum $c_i$.
By requiring that $s_i + c_i = z$ and using
the fact that $s_i$ is positive
($\Xb$ is semidefinite so $S$ is as well),
we can keep $z$ larger than or equal to $c_i$.
The semidefinite program minimizes $z$
by our choice of $C$ so $z$ meets the maximum $c_i$.

To enforce $s_i + c_i = z$, define $\Ap_i$ as a diagonal
matrix  with 1s along the chunk
corresponding to the $i^{th}$ input,
1 in the entry corresponding to $s_i$,
and -1 in the entry corresponding to $z$.
In a slight abuse of notation, we show an example
of $\Xb A_i$.
\begin{align}
\left[\begin{matrix} c_i & 0 & 0 \\
                    0 & s_i & 0 \\
                    0 & 0 & z \end{matrix} \right]
\left[\begin{matrix} 1 & 0 & 0 \\
                    0 & 1 & 0 \\
                    0 & 0 & -1 \end{matrix} \right]
= \left[\begin{matrix} c_i & 0 & 0 \\
                    0 & s_i & 0 \\
                    0 & 0 & -z \end{matrix} \right]
            \nonumber
\end{align}
Notice that the trace of $\Xb A_i$
is $c_i + s_i - z$.
By setting $b_i = 0$, we have that $z$ is
greater than or equal to $c_i$.

The final matrix we define is $C$.
Our goal is to select out $z$ from $\Xb$
so $C$ has a single non-zero value in the 
entry corresponding to $z$.
\begin{align}
    C = \left[\begin{matrix} 0  \cdots 0 \\
                                \vdots \ddots \vdots \\
                                0 \cdots  1  \end{matrix}
                                \right]  
                                \nonumber
\end{align}

We will now prove that using our construction,
$z = \max\{c_1, c_2, ... c_{|D|}\}$ in the optimal $\Xb$

Notice that because our matrix $\Xb$ is semidefinite 
and diagonal, all of its diagonal entries are non-negative. 
Assume for contradiction that $z < c_i$
for some $i \in \{1,2,..., |D|\}$.
Therefore $z < c_i + s_i$ since $s_i \geq 0$.

By our constraint, $z = c_i + s_i$, 
meaning that $z < z$, a contradiction! 
Therefore it must be the case that $z$ is an 
upper bound of the traces of our sub-matrices $c_i$. 

Now assume for contradiction that $z \ne  \max\{c_1, c_2, ...
c_{|D|}\}$. We know that $z$ is an upper bound so we only need to
show that there is no upper bound of this finite set that is less
than $z$ to show it is the maximum. Our assumption is equivalent to
the statement that there does not exist a constant $a$ such that $a
> c_i \forall i$ and $a < z$. If $a$ did exist, then there would
exist an $\Xb$ that also satisfies all of our constraints,
but with $a$ in the bottom right most entry instead of $z$. This
would mean that our objective function was improved over our
initial matrix as $a < z$, meaning that the $\Xb$ we found
is not optimal: a contradiction. We conclude that our constraints
are equivalent to those outlined by Reichardt, while matching the
form of Boyd.

Once given a function $f$ and all possible inputs to the
function of length $n$ ($D$), we can then define matrices $C$ and
$A_i$, as well as scalars $b_i$ such that the solution of this
problem produces the solution of the original problem which
produces the optimal bound of quantum query complexity for
arbitrary Boolean functions.