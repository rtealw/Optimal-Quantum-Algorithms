\section{Methodology}\label{sec:method}

\subsection{Original Form of the SDP}
Consider a Boolean function $f$ such that
$f: D \rightarrow E$ where 
$D \subseteq {\{0,1\}}^n$ and $E \subseteq {\{0,1\}}$.
We state Reichardt's formulation of the SDP \cite{reichardt2009span}.
(Note that $[n]$ denotes the array
$[1,2,...,n]$.)
The quantum query complexity of $f$ is
\begin{align} \label{Eq:reichardt_obj} 
    \max_{y \in D} \sum_{j \in [n]} \bra{y,j}\X\ket{y,j} 
\end{align}
for some $\X$ subject to the constraints that $\X$ is positive semidefinite
(all its eigenvalues are non-negative)
and for $y$ and $z$ such that $f(y) \neq f(z)$,
\begin{align}\label{Eq:reichardt_F0}
    \sum_{j \in [n]: y_j \ne z_j} 
    \bra{y,j} \X \ket{z, j} = 1.
\end{align}

\subsection{Standard Form of the SDP}
In order to implement an algorithm that solves 
Reichardt's formulation, we convert the SDP
into the canonical standard form
\cite{boyd2004convex}.
Then the standard form objective function of the SDP is
\begin{align}\label{Eq:boyd_obj}
    \tr(C\Xb)
\end{align}
for some fixed matrix $C$ subject to the constraints
that $\Xb$ is semidefinite and for $i \in [p]$
\begin{align} \label{Eq:boyd_trace}
    \tr(A_i \Xb) = b_i
\end{align}
for some fixed matrices $A_i$ and $b_i$.

\subsection{Converting from Original to Standard Form}
We now convert the original form of the SDP to
the standard form by constructing appropriate matrices $A_i$, $b_i$, $\Xb$, and $C$.

We start by gaining some intuition about $\X$.
It is clear from \cref{Eq:reichardt_obj} that for 
each $y \in D$, there exists an $n \times n$ sub-matrix of $\X$ along its diagonal.
Then $\X$ must have a total of $n|D|$ rows and $n|D|$ columns.
(In general, $|D| = 2^n$ but it is possible to take 
worst-case subsets of $D$ and still find the correct solution.)

To convert \cref{Eq:reichardt_F0} into \cref{Eq:boyd_trace},
we must construct $A_i$ for $i \in [p]$.
Let $p = |F|$, then each pair $(y,z) \in F$
is associated with a single $A_i$.
For $(y,z)$, we are interested in the entries
in $\X$ that correspond to the bits $y_i$ and $z_i$
where $y_i \neq z_i$ for $i \in [n]$.
These entries must necessarily be off of the main diagonal,
because $y\neq z$,
so our $A_i$ will map them to the main diagonal 
and ensure that the trace of $A_i \X$ is 1.
Note that $b_i = 1$.

To place particular entries of $\X$ onto the main diagonal,
let $A_i$ be the transpose of the zero matrix with
ones in the particular entries of interest.
We illustrate on a small example.
Let
\begin{align}
    \X = \left[ \begin{matrix} 1 & 2 \\ 3 & 4 \end{matrix} \right] \nonumber
\end{align}
and the particular entry of interest be the top right value of $\X$.
Then
\begin{align}
    A_i = \left[ \begin{matrix} 0 & 1 \\ 0 & 0 \end{matrix} \right]^\textrm{T} \nonumber
    = \left[ \begin{matrix} 0 & 0 \\ 1 & 0 \end{matrix} \right] \nonumber
\end{align}
so that
\begin{align}
    A_i \cdot \X = \left[ \begin{matrix} 2 & 0 \\ 4 & 0 \end{matrix} \right] \nonumber
\end{align}
and $\tr(A_i \cdot \X) = 2$ as desired.

Let $c_i$ be the sum of the $n$ entries
in the $i^\mathrm{th}$ section of the main diagonal of $\X$
such that
\begin{align}
    c_i = \sum_{j \in [n]} \bra{y,j}\X\ket{y,j}
    \nonumber
\end{align} 
where $y$ is the $i^{th}$ element of $D$ for
$i \in [|D|]$.
Let $z$ be $\tr(C\Xb)$ and, so as to extract the bottom right entry,
let $C$ be the zero matrix with a single one in the bottom right entry.
The last challenge is to satisfy \cref{Eq:boyd_obj}
by enforcing that $z$ is the maximum $c_i$.
Our solution is to introduce non-negative slack variables
$s_i$ for $i \in [|D|]$ and the constraints
$c_i + s_i = z$.

To construct $\Xb$, call $S$ the zero matrix
whose diagonal holds the entries $s_1,...,s_{|D|}, z$.
Then
\begin{align}
    \Xb = \left[ \begin{matrix} \X & 0 \\ 0 & S \end{matrix} \right] \nonumber.
\end{align}
Since $\Xb$ is semidefinite, the main diagonal entries of
$\Xb$ must be non-negative so $s_i \geq 0$.

To enforce that $z$ is the maximum $c_i$ over all $i$,
we construct $\Ap_i$ and set $\bp_i$ to 0
for $i \in [|D|]$. For the purposes of the standard form,
we now have two sets of $A_i$ and $b_i$ constraints.
Let $\Ap_i$ be the zero matrix with ones
along the entries corresponding to the input $i$ of $D$,
a one in the entry corresponding to $s_i$,
and a negative one in the bottom right entry corresponding
to the objective function $z$.
Then
\begin{align}
    \tr (\Ap_i \Xb ) &= c_i + s_i -z = 0 \nonumber.
\end{align}
Because $s_i$ is non-negative and $z$ is being minimized,
$z$ is both greater than or equal to all $c_i$ and meets
some $c_i$. Thus $z$ is the maximum $c_i$ as desired.

The solution $\Xb$ that satisfies the standard form according
to the constructed $A_i$, $\Ap_i$, $b_i$, $\bp_i$, and $C$ will 
also hold the solution $\X$ to the original form.
Therefore, to find the optimal quantum query complexity of $f$,
we need only solve \cref{Eq:boyd_obj} subject to our constraints.

\subsection{Query Efficient Span Programs}

A span program is a model of computation that outputs a
Boolean value corresponding to whether or not a set
of vectors "spans" to a target vector. The algorithm consists
of a set of
vectors such that on a given input, the vectors are
partitioned into available and unavailable sets. The
available vectors are turned into the columns of a
matrix $A$.
These vectors form a linear system $Ax = \tau$
for a fixed target vector $\tau$.
If this system is homogeneous--and there exists
some non-trivial vector $x$ that satisfies the equation--
then the algorithm returns true.
Otherwise, the algorithm returns false.

Both Childs and Reichardt mathematically formulate
algorithms for turning the results of the SDP we solved
into a span program \cite{reichardt2009span, childs}.
To begin, we must construct a set of vectors
$\bra{v_{y,i}}$ for all $y \in D$ where $i = [n]$.
Given any two input strings $y, z \in D$ such that $y\neq z$,
\begin{align}
    \sum_{i: y_i \ne z_i} \braket{v_{y,i}|v_{z,i}} = 1 - \delta_{f(y), f(z)} \nonumber
\end{align}
where $\delta_{f(y), f(z)} = 1$ if $f(y) = f(z)$ and $0$ otherwise.

Recall from \cref{Eq:reichardt_F0} that $\X$ satisfies
\begin{align}
    \forall (y,z) \in F \sum_{j \in [n]: y_j \ne z_j} 
    \bra{y,j} \X \ket{z, j} = 1 \nonumber.
\end{align}
We would like to construct $\bra{v_{y,i}}$ and $\ket{v_{z,i}}$
from $\X$.
So we define a matrix $L$ such that $L^\dagger L = \X$.
Then $\bra{v_{y,i}} = \bra{y_i}L^\dagger$.
We can now reformulate the requirement of vectors $\bra{v_{y,i}}$
as
\begin{align}
    \sum_{i: y_i \ne z_i} \braket{v_{y,i}|v_{z,i}} &= \sum_{i: y_i \ne z_i}
    \bra{y,i} L^\dagger L \ket{z, i} \nonumber \\
    &= \sum_{i: y_i \ne z_i} \bra{y,i} \X \ket{z, i} = \begin{cases}
        1 & f(y) \ne f(z) \\
        0 & f(y) = f(z)
    \end{cases} \nonumber 
\end{align}
The constraints enforce the $f(y) \neq f(z)$ case and we
check the $f(y) = f(z)$ case before returning $\X$.

Given the set of vectors $\bra{v_{y,i}}$ for all $y \in
D$, $i = 0, 1, \ldots , n-1$, we construct 
matrix $I$ that contains the vectors for the span
program. We will make use of the sets $F_i =
\{y \in D: f(y) = i\}$.
(Note that $F_i[k]$ denotes the $k^{th}$ element of $F_i$.)

\begin{align}\label{input_vectors}
    I = \sum_{k \in [|F_0|], j \in [n], y = F_0[k]}
    \ket{k}\bra{j, \overline{y_j}} \bra{v_{y,j}}
\end{align}

This matrix $I$ is divided into sub-matrices that
correspond to different bits in the input to the
algorithm.
The columns of $I$ are evenly divided into
$n$ chunks corresponding to each input bit.
Each of these sub-matrices is further divided into a left and
right sub-matrix, corresponding to a $0$ or $1$
in the relevant bit of the input.
Let $I(y)$ be the set of columns in 
$I$ that are available on input $y \in D$.
The target vector $\tau$ 
is a vector of ones vector with the same 
dimension as $I$.
The output is true if and only if the available
vectors $I(y)$ span to $\tau$.
Reichardt proves that the span program $I$ 
and $\tau$ is a query optimal 
quantum algorithm for $f$ \cite{reichardt2009span}.
