\section{Methodology}

Consider the function $f: D \rightarrow E$ where $D
\subseteq {\{0,1\}}^n$ and $E \subset {0,1}$. We aim to find an
optimal bound on quantum query complexity using semi-definite
programming. Using existing research
\cite{reichardt2009span} we can formulate the problem as below, where
$f_{\text{bound}}$ represents the optimal bound for the function
$f$ with the input size $n$.
Let $F$ be the set of $(y,z)$ such that $f(y) \neq f(z)$.
We then define this optimization problem as:
\begin{align} \label{Eq:reichardtObj}
    M(\X) = \max_{y \in D} \sum_{j \in [n]}
    \bra{y,j}\X\ket{y,j} 
\end{align}
subject to
\begin{align}\label{Eq:semi1}
    \X \succcurlyeq 0 
\end{align}
and
\begin{align}\label{Eq:off-diag}
    \forall (y,z) \in F \sum_{j \in [n]: y_j \ne z_j} 
    \bra{y,j} \X \ket{z, j} = 1.  
\end{align}

where we aim to minimize the value of $M$ by finding the optimal value of $\X$.

To be clear, we illustrate the operations used in these definitions. Consider equation \cref{Eq:reichardtObj}, observing that $\mathbb{X}$ is a $n2^n$ x $n2^n$ matrix because there are $2^n$ possible inputs, each of length $n$. Now represent $\mathbb{X}$ as containing chunks of size $n$ x $n$, where each chunk corresponds to an element of $D$ x $D$. Below we again consider the OR function with one bit inputs.

\begin{align}
   X = \left[ \begin{matrix}
   \left[ \begin{matrix} (0,0)\end{matrix} \right] & \left[
   \begin{matrix} (0,1)\end{matrix} \right] \\ \\
   \left[ \begin{matrix} (1,0)\end{matrix} \right] & \left[
   \begin{matrix} (1,1)\end{matrix} \right] \\
    \end{matrix} \right]
\end{align}

Our objective function (eq. \cref{Eq:reichardtObj}) is
the maximum trace of the diagonal sub-matrices. The
first constraint is that $\mathcal{X}$ is positive
semi-definite. The final constraint states that the
trace of each sub-matrix, for sub-matrices that
correspond to two inputs with different outputs (e.g.,
(0,1) and (1,0)), must sum to one, only considering
elements along the diagonal that correspond to bits in
the two input strings that differ. 

Given this non-traditional formation of the semi-definite programming problem, we must first convert the problem into an equivalent problem in standard form as defined by Boyd \cite{boyd2004convex}:
\begin{align}
    M(\X) = tr(C\X) 
\end{align}
subject to
\begin{align}\label{Eq:semi2}
    \X \succcurlyeq 0   
\end{align}
and for $i \in \{1,...,|F|\}$
\begin{align}\label{Eq:trace}
    tr(A_i \X) = b_i. 
\end{align}


To make these two problems equivalent we first note that
conditions \cref{Eq:semi1} and \cref{Eq:semi2} are
equivalent.
To equate conditions \cref{Eq:off-diag}
and \cref{Eq:trace}, we must define the set of matrices
$A_i$ and
scalars $b_i$. Naturally, the value of $b_i$ shall be $1$
from \cref{Eq:off-diag}.
Observe that the sum in condition \cref{Eq:off-diag}
iterates through a portion of the matrix $\X$ diagonally.
Thus we can define a set of matrices where the matrix 
$A_i$ contains only ones and zeros
such that $A_i \X$ puts the off-diagonal values we want
to sum onto the diagonal.
Therefore the trace of this new matrix $A_i \X$ 
will be equal to the sum in \cref{Eq:off-diag}.

We will then take our original $\X$ and define a new
$\mathcal{X}$:
\begin{equation}
    \mathcal{X} = \left[\begin{matrix}\X \cdots 0 \\
                                \vdots \ddots \vdots \\
                                0 \cdots z  \end{matrix}
                                \right]  
\end{equation}

\qquad Once given a function $f$ and all possible inputs to the
function of length $n$ ($D$), we can then define matrices $C$ and
$A_i$, as well as scalars $b_i$ such that the solution of this problem
produces the solution of the original problem which produces the
optimal bound of quantum query complexity for the given quantum
algorithm.