\section{Methodology}

Consider the function $f: D \rightarrow E$ where $D
\subseteq {\{0,1\}}^n$ and $E \subset {0,1}$. We aim to find an
optimal bound on quantum query complexity using semi-definite
programming. Using existing research
\cite{reichardt2009span} we can formulate the problem as below, where
$f_{\text{bound}}$ represents the optimal bound for the function
$f$ with the input size $n$.
Let $F$ be the set of $(y,z)$ such that $f(y) \neq f(z)$.
We then define this optimization problem as:
\begin{align}
    M(\X) = \max_{y \in D} \sum_{j \in [n]}
    \bra{y,j}\X\ket{y,j} \nonumber
\end{align}
subject to
\begin{align}\label{Eq:semi1}
    \X \succcurlyeq 0
\end{align}
and
\begin{align}\label{Eq:off-diag}
    \forall (y,z) \in F \sum_{j \in [n]: y_j \ne z_j} 
    \bra{y,j} \X \ket{z, j} = 1. 
\end{align}

where we aim to minimize the value of $M$ by finding the optimal value
of $\X$. To make use of Boyd's classic book on convex 
optimization \cite{boyd2004convex} we can then reform this 
problem into classic semi-definite programming form:
\begin{align}
    M(\X) = tr(C\X) \nonumber
\end{align}
subject to
\begin{align}\label{Eq:semi2}
    \X \succcurlyeq 0  
\end{align}
and for $i \in \{1,...,|F|\}$
\begin{align}\label{Eq:trace}
    tr(A_i \X) = b_i. 
\end{align}


To make these two problems equivalent we first note that
conditions \cref{Eq:semi1} and \cref{Eq:semi2} are equivalent.
To equate conditions \cref{Eq:off-diag}
and \cref{Eq:trace}, we must define the set of matrices $A_i$ and
scalars $b_i$. Naturally, the value of $b_i$ shall be $1$
from \cref{Eq:off-diag}.
Observe that the sum in condition \cref{Eq:off-diag}
iterates through a portion of the matrix $\X$ diagonally.
Thus we can define a set of matrices where the matrix 
$A_i$ contains only ones and zeros
such that $A_i \X$ puts the off-diagonal values we want
to sum onto the diagonal.
Therefore the trace of this new matrix $A_i \X$ 
will be equal to the sum in \cref{Eq:off-diag}.

We will then take our original $\X$ and define a new $\mathcal{X}$:
\begin{equation}
    \mathcal{X} = \left[\begin{matrix}\X \cdots 0 \\
                                \vdots \ddots \vdots \\
                                0 \cdots z  \end{matrix} \right]  \nonumber
\end{equation}

\qquad Once given a function $f$ and all possible inputs to the
function of length $n$ ($D$), we can then define matrices $C$ and
$A_i$, as well as scalars $b_i$ such that the solution of this problem
produces the solution of the original problem which produces the
optimal bound of quantum query complexity for the given quantum
algorithm.