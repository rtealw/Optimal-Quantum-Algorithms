\section{Introduction}

Given the vast potential of quantum computers to solve
computationally difficult problems more efficiently
than classical computers, it is of interest to
determine for which functions this is the case. Although a
simple solution would be to run equivalent functions using
both quantum and classical computers and compare their
performance, sufficiently large quantum computers don't
currently exist.

Therefore, we must rely on analytical, asymptotic methods
to characterize the run time on quantum computers on
complex functions. More specifically, we make use of the
query complexity of a function on a quantum computer to
place a lower bound on its run time. Simply put, the query
complexity is the number of times we must examine the
contents of the input string to determine the correct
output of the function. This represents a lower bound of
the asymptotic run time of the optimal algorithm because to
evaluate a function, we must examine the input string the
minimum number of times and additionally perform other
computations to evaluate the function. These other
computations could overtake the queries of the input string
to alter the overall run time of the algorithm, meaning the
run time is at least as large as the query complexity. In comparing quantum and classical algorithms for evaluating a function, we conclude that the algorithm with the lower query complexity is more efficient.

To give a concrete example for query complexity, as well as
for the remainder of the paper, we invite the reader to
recall the canonical one bit OR function. This function
returns a $1$ if at least one of the inputs is $1$ and
otherwise results in a $0$. More generally, the $n$-bit OR
function returns $0$ if there are only $0$'s in the input
string and $1$ if there are any $1$'s in the input string.
Classically, an algorithm must check that
every single bit of the input is a $0$ to return $0$.
Therefore the classical query complexity of the $n$-bit OR
function is $n$. However, the quantum algorithm Grover's
search can solve the function OR in $\sqrt{n}$ queries
\cite{grover1996fast}. In this instance, we conclude that
the quantum evaluation of the $n$-bit OR function is more
efficient than the classical evaluation.

Given an arbitrary Boolean function, we can calculate its
asymptotic quantum query complexity analytically for
comparison with known classical query complexity. More
specifically, we focus on optimal adversary bounds of
quantum query complexity. This approach calculates an upper
bound on quantum query complexity for a function over all
inputs by finding the complexity of the worst-case input.
Lee et al. show that this optimal upper bound of quantum
query complexity corresponds to the solution of a
semidefinite program (referred to as SDP)
\cite{lee2011quantum}. Reichardt thenreformulated this SDP
into a form we found more useful in our analysis
\cite{reichardt2009span}. We convert Reichardt's
form into the standard SDP definition as defined by Boyd
\cite{boyd2004convex}. We then implement an SDP
solver to find the quantum query complexity of arbitrary
Boolean functions, noting that although this SDP
formulation is valid for arbitrary functions, we limit the
scope of our implementation to Boolean functions.

There exist many implementations of SDP solvers in Python
libraries for convex optimization, such as CVXOPT and SDPA
\cite{cvxopt, SDPA}. However, neither of these libraries,
or others we examined, easily support the standard form of
the SDP problem we have formulated. We implement our own
solver in Python to specifically target the solution of
SDP's corresponding to these bounds on quantum query
complexity. Our algorithm is an alternating
direction method (ADM). We made this choice because Wen et
al.'s algorithm purports to efficiently exploit the
sparsity of SDP's --- a structure we frequently find in our
formulation of these problems \cite{adm}.

Mathematically, our goal is to create a tool that takes as
input a natural number $n$ and a Boolean function 
$f: \{0,1\}^n \rightarrow \{0,1\}$. 
By solving Reichardt's SDP problem with
Wen et al.'s ADM algorithm, we return the optimal adversary bound for the quantum query complexity of $f$. We hope that quantum algorithm researchers use our tool to discover and verify the optimality of quantum algorithms.