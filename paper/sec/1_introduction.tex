\section{Introduction}

We consider Boolean functions where inputs
are a binary string and whose outputs are a single bit.
For example, let $f$ be the "or" function.
Then $f$ takes as input $x \in \{0,1\}^n$ and outputs
0 if $x$ contains no 1's and 0 otherwise.
The semi-definite program we plan to implement
takes a Boolean function and a number $n$, which is the
length of the input bit strings, and returns the most
number of queries the optimal quantum
algorithm would need to evaluate the function.
An algorithm performs a query every time it checks the value
of an input.
For example, a classical algorithm would need to check every
bit in $x$ before concluding that no 1's are present.
However, a quantum algorithm may use Grover's search
to find a 1 in $\sqrt{n}$ queries.

There currently exists a robust literature regarding the
optimal adversary bound of quantum query complexity as
the solution to a semi-definite programming problem
\cite{beigi2018span}. This formulation is available in
prior research and we consider this axiomatically,
leaving this proof to the reader. The form we assume for
this problem is provided by Reichardt
\cite{reichardt2009span}. We then reformulate the problem
to conform to the standard semi-definite programming form
as defined by Boyd \cite{boyd2004convex}. Given the
exponential nature of the storage and computation of the
SDP, we hope to exploit some structure in the problem.

It seems that the literature is still lacking and
implementation of an
SDP solver that leverages the unique structure found in
these problems. We hope to improve on existing approaches
to solving these convex optimization problems while
creating a more thorough understanding of these
algorithms.