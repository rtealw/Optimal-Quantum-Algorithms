\section{Conclusion}
Our algorithm correctly identifies the optimal
quantum query complexity for the Boolean functions
we are familiar with, as well as produces valid span
programs for each function we have tested. Our aim is that this program could be a tool in identifying and implementing functions and algorithms with 
quantum speed-ups, which is an important area
of quantum computing research.

While the SDP solution doesn't necessarily provide an immediate use to researchers, as we are unfamiliar with Boolean functions with known optimal quantum query complexity, we hope that it can be used as an aid for verifying existing understand. Furthermore, our production of the query optimal span program will hopefully prove helpful in the development of
optimal quantum algorithms and gaining intuition into optimal quantum algorithms.

Although there is much our program has accomplished, there are several possible avenues for future work. Additional functions with worst-case
inputs may be explored to reduce runtime as seen with OR. We could also expand the class of functions our algorithm can address as Reichardt's proofs demonstrate that a similar approach can be used for functions that map to any finite set of outputs.
Once the SDP is solved, it could be helpful to then have a feature to identify the best polynomial
fit (e.g. squareroot or linear) toautomatically categorize the asymptotic behavior
of a given Boolean function. Our solver currently solver each bitstring length and presents the results visually, but the asymptotic behavior could be classified algorithmicallty.
Finally, further optimizations in the
eigenvalue decomposition step of our iteration
can be made to further reduce runtime.

\begin{acks}
We would like to thank Professors Shelby Kimmel
and Philip Caplan for their invaluable help and support. 
\end{acks}