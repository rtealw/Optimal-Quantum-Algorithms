\documentclass[acmtog]{acmart}
%acmtog if we're feeling a little funky ;)

\usepackage{booktabs, braket} % For formal tables
\usepackage{cleveref}
\usepackage{graphicx}
\usepackage{blkarray}

\settopmatter{printacmref=false} % Removes citation info after abstract
\renewcommand\footnotetextcopyrightpermission[1]{} % removes footnote with conference informat\textbf{}ion in first column
\usepackage{fancyhdr}

\DeclareMathOperator{\X}{\mathbb{X}}		     % expected value
\DeclareMathOperator{\tr}{\text{tr}}		     % trace
\DeclareMathOperator{\Ap}{$A$ ^\prime} 		     % A'
\DeclareMathOperator{\bp}{$b$ ^\prime} 		     % b'
\DeclareMathOperator{\Xb}{\mathcal{X}}		     % big X
 
\pagestyle{fancy}
\fancyhf{}
\lhead{Czekanski \& Witter, Fall 2019}
\rfoot{\thepage}
\newcommand{\todo}[1]{{\color{red}{[{\bf TODO:} #1]}}}

%\pagestyle{plain} % removes running heades

% Use the "authoryear" citation style, and make sure citations are in [square brackets].
\citestyle{acmauthoryear}
\setcitestyle{square}

% A useful command for controlling the number of authors per row.
% The default value of "authorsperrow" is 2.
\settopmatter{authorsperrow=1}

\begin{document}


% Title. 
% If your title is long, consider \title[short title]{full title} - "short title" will be used for running heads.
\title{Semidefinite Program Solver to 
Find the Optimal Quantum Query Complexity
and Query Optimal Quantum Algorithm
of Boolean Functions}

% please leave the subtitle!
\subtitle{Middlebury College, Fall 2019}

% Authors.
\author{Michael Czekanski}
\affiliation{%
  \institution{Middlebury College}}

\author{R. Teal Witter}
 \affiliation{%
   \institution{Middlebury College}}

\maketitle

\graphicspath{{./../figures/}}
% a couple of paragraphs describing your project
\section*{Abstract}

With the increasing size and computational power of quantum
computers, we want to find the problems that quantum computers
can solve more efficiently than classical computers.
It's essential to identify optimal quantum algorithms for this
comparison. We define an optimal quantum algorithm to be the
algorithm that solves a given problem with the lowest
asymptotic query complexity. It's also important to consider
post-processing steps and overall runtime, but we consider a
query model where an algorithm that evaluates the Boolean
function $f$ takes as input a bitstring and returns 0 or 1
by simply querying the input. For a given function $f$, the
optimal query complexity of $f$ is the query complexity of the
query optimal algorithm. The optimal query complexity for $f$
is the worst-case number of queries the best
quantum algorithm needs to solve $f$.
The query optimal algorithm for $f$ is
the algorithm that evaluates $f$ and
meets the optimal query complexity.
In this paper, we implement a semidefinite program
whose solution gives both the optimal quantum query
complexity and a query optimal quantum algorithm.

We call our Python package QuantumQueryOptimizer
and hope researchers will find it helpful
in gaining intuition about how to construct optimal
quantum algorithms.

% the other sections
\section{Introduction}

Quantum computers have the potential to 
solve computationally difficult problems
faster and more efficiently than classical computers.
Since sufficiently large quantum computers do not yet exist,
we cannot compare the time it takes a quantum computer
to run an algorithm to the time it takes a classical computer
to run an equivalent algorithm.
Instead we rely on asymptotic run time that 
characterizes the time an algorithm will take in relation
to the size of the input.
In the study of quantum algorithms, it is often easier
and more elegant to describe an algorithm's query complexity.
Imagine a model where an algorithm is given access
to an input as a string of bits.
The query complexity of an algorithm is then the number
of bits the algorithm needs to reveal to determine a solution.
Query complexity is a lower bound for run time
(each query takes one time step);
however, run time can asymptotically outpace query complexity.

Consider the Boolean function OR.
An algorithm for the OR function returns 0 if
there are only 0's in the input string
and 1 if there are any 1's in the input string.
Classically, an algorithm must check that
every single bit of the input is a 0.
Then if there are $n$ bits in the input string
the classical algorithm's query complexity is $n$.
However, the quantum algorithm Grover's search
can solve the function OR in $\sqrt{n}$ queries \cite{grover1996fast}.

In the study of quantum algorithms,
we call the query complexity of the most efficient quantum
algorithm optimal.
Then when comparing quantum and classical algorithms,
we can say that the quantum algorithm is more
efficient than the classical algorithm in the sense
that the optimal quantum query complexity
is lower than the optimal classical query complexity.

Lee et al. show that optimal quantum query complexity
corresponds to the solution of a semidefinite program (SDP)
\cite{lee2011quantum}.
Reichardt reformulates the SDP into a more simple description 
\cite{reichardt2009span}.
We take Reichardt's formulation and convert it
to a standard definition found in Boyd \cite{boyd2004convex}.
We then implement an SDP solver to find the
quantum query complexity of arbitrary Boolean functions.
(While the SDP can solve any function,
we limit our scope to Boolean functions with
binary inputs and outputs).

CVXOPT and SDPA are both popular, existing convex optimization
libraries in Python \cite{cvxopt, SDPA}.
However, neither libraries easily support the standard form
of the SDP problem we are solving.
We instead turn to an alternating direction method (ADM).
Wen et al.'s algorithm efficiently exploits the 
sparsity of our SDP \cite{adm}.

Our goal is to create a tool that takes as input a 
natural number $n$ and a Boolean function 
$f: \{0,1\}^n \rightarrow \{0,1\}$. 
By solving Reichardt's SDP problem with
Wen et al.'s ADM algorithm,
we return the optimal quantum query complexity of
$f$. We hope that quantum algorithm researchers use
our tool to discover and verify optimal quantum
algorithms.
\section{Methodology}

Consider the function $f: D \rightarrow E$ where $D
\subseteq {\{0,1\}}^n$ and $E \subset {0,1}$. We aim to find an
optimal bound on quantum query complexity using semi-definite
programming. Using existing research
\cite{reichardt2009span} we can formulate the problem as below, where
$f_{\text{bound}}$ represents the optimal bound for the function
$f$ with the input size $n$.
Let $F$ be the set of $(y,z)$ such that $f(y) \neq f(z)$.
We then define this optimization problem as:
\begin{align}
    M(\X) = \max_{y \in D} \sum_{j \in [n]}
    \bra{y,j}\X\ket{y,j} \nonumber
\end{align}
subject to
\begin{align}\label{Eq:semi1}
    \X \succcurlyeq 0
\end{align}
and
\begin{align}\label{Eq:off-diag}
    \forall (y,z) \in F \sum_{j \in [n]: y_j \ne z_j} 
    \bra{y,j} \X \ket{z, j} = 1. 
\end{align}

where we aim to minimize the value of $M$ by finding the optimal value
of $\X$. To make use of Boyd's classic book on convex 
optimization \cite{boyd2004convex} we can then reform this 
problem into classic semi-definite programming form:
\begin{align}
    M(\X) = tr(C\X) \nonumber
\end{align}
subject to
\begin{align}\label{Eq:semi2}
    \X \succcurlyeq 0  
\end{align}
and for $i \in \{1,...,|F|\}$
\begin{align}\label{Eq:trace}
    tr(A_i \X) = b_i. 
\end{align}


To make these two problems equivalent we first note that
conditions \cref{Eq:semi1} and \cref{Eq:semi2} are equivalent.
To equate conditions \cref{Eq:off-diag}
and \cref{Eq:trace}, we must define the set of matrices $A_i$ and
scalars $b_i$. Naturally, the value of $b_i$ shall be $1$
from \cref{Eq:off-diag}.
Observe that the sum in condition \cref{Eq:off-diag}
iterates through a portion of the matrix $\X$ diagonally.
Thus we can define a set of matrices where the matrix 
$A_i$ contains only ones and zeros
such that $A_i \X$ puts the off-diagonal values we want
to sum onto the diagonal.
Therefore the trace of this new matrix $A_i \X$ 
will be equal to the sum in \cref{Eq:off-diag}.

We will then take our original $\X$ and define a new $\mathcal{X}$:
\begin{equation}
    \mathcal{X} = \left[\begin{matrix}\X \cdots 0 \\
                                \vdots \ddots \vdots \\
                                0 \cdots z  \end{matrix} \right]  \nonumber
\end{equation}

\qquad Once given a function $f$ and all possible inputs to the
function of length $n$ ($D$), we can then define matrices $C$ and
$A_i$, as well as scalars $b_i$ such that the solution of this problem
produces the solution of the original problem which produces the
optimal bound of quantum query complexity for the given quantum
algorithm.
\section{Results}

We will make progress :)
\section{Changes to Scope}

The most obvious next step is to
speed up our SDP solver.
The slowest part of our implementation is the
process of decomposing a very large matrix at
each iteration.
While we will always need to implement
such a computationally expensive step,
we may save space and time by optimizing other 
components of our implementation.
We could increase the efficiency that
we store our matrices or otherwise 
decrease the computational work done at each step.
However, even in the best case where
we double or triple the speed of our implementation,
we are hitting the exponential size of the SDP.
A scalar improvement will not make a meaningful
difference in our ability to solve different sizes
of SDP.
Right now, we can solve SDPs on all bit strings
of length 4 or maybe 5 in reasonable time
but we will not be able to reasonably solve
SDPs with all 6 or 7 bit strings no matter the extent
to which we optimize our implementation.

Computationally we cannot make much progress.
But if we can limit the inputs to our SDP,
only considering the worst-case bit strings
of the Boolean function,
we can decrease the size of our SDP problem
and therefore the time it takes to solve.
As demonstrated in \cref{sec:speed}, 
the strategy of considering only the worst-case
inputs substantially improves the speed of solving the SDP.
One way we can limit the Boolean inputs is from
insights about the Boolean function.
We have demonstrated this approach in perhaps the most
trivial case with OR.
We can extend this strategy by considering additional
Boolean functions.
The other, more general way we can limit the number
of Boolean inputs is by optimizing our SDP solver.
Perhaps within a few iterations, we can tell
that certain inputs take relatively few queries to find
the output.
By optimizing as we go, we can apply our strategy to
an Boolean function while simultaneously cutting down
the time run time of solving it.

Another direction for our work is to implement
a distinct but related problem.
Our SDP solver solves the primal:
finding the optimal quantum query complexity
of arbitrary Boolean functions)
There also exists a dual SDP whose solution 
corresponds to the optimal quantum algorithm itself.
We can implement the dual of our current problem
and, in addition to finding the query complexity,
also find the algorithm itself.



\begin{acks}
The authors would like to thank Professors Shelby Kimmel
and Philip Caplan for their invaluable help and support. 
\end{acks}

% select the reference format (leave this)
\bibliographystyle{ACM-Reference-Format}

% specify the bibliography (.bib) file
% place your references in this bib file
\bibliography{main}

\end{document}